\documentclass{article}
\usepackage{amsmath}
\usepackage[UTF8, fontset=none]{ctex} % 禁用默认字体
\usepackage{fontspec}
\setmainfont{Noto Sans CJK SC} % 使用Overleaf预装的Noto字体
\begin{document}

假设地面上两个天线的位置分别为$\vec{x}_1$和$\vec{x}_2$,天空中射电源的方向由方向余弦$\vec{n}_i$描述,其频谱为$I_i(\nu)$。则以频率$f$为中心、极小带宽内,两个天线感应电压$V_1(f)$和$V_2(f)$的互相关。
\\

\textbf{解答:}
由 $E(V^2) \propto I(\vec{n},\nu)d\Omega$,$V \propto E(\vec{x},t) \propto  I(\vec{n},\nu)^{1/2} d\Omega $ 得:

天线感应电压表达式:
\begin{equation}
    V(\mathbf{n}, t,\nu) = K \int I^{1/2}(\mathbf{n}, \nu) e^{i2\pi \nu(t- \frac{\mathbf{n} \cdot \mathbf{x}}{c})} \, d\nu \, d\Omega
\end{equation}
其中:
\begin{itemize}
    \item $K$ 为比例系数
    \item $I(\mathbf{n}, \nu)$ 是方向 $\mathbf{n}$ 的强度
    \item $\mathbf{x}$ 为天线位置
\end{itemize}

当 $\Delta \nu \to 0$ 即以 $f$ 为频率中心时,感应电压简化为:
\begin{equation}
    V(\mathbf{n}, f) = K \sqrt{\Delta \nu} \int I^{1/2}(\mathbf{n}, f) e^{i2\pi f(t- \frac{\mathbf{n} \cdot \mathbf{x}}{c})}\, d\Omega
\end{equation}
其中 $\sqrt{\Delta \nu}$ 源于频带积分。

假定信号同时到达 $\delta t \to 0$,$V_1$ 和 $V_2$ 的互相关为:
\begin{equation}
    \langle V_1 V_2^* \rangle = K^2 \Delta \nu \iint \langle I^{1/2}(\mathbf{n},f)I^{1/2}(\mathbf{n}',f) \rangle e^{i2\pi \frac{(\mathbf{n} \cdot \mathbf{x}_1 - \mathbf{n}' \cdot \mathbf{x}_2)}{\lambda}} \, d\Omega d\Omega'
\end{equation}
其中 $\lambda = c/f$。

由于不同方向的源不相关,有:
\begin{equation}
    \langle I^{1/2}(\mathbf{n})I^{1/2}(\mathbf{n}') \rangle = I(\mathbf{n}) \delta(\mathbf{n}-\mathbf{n}')
\end{equation}
积分简化为:
\begin{equation}
    \langle V_1 V_2^* \rangle = K^2 \Delta \nu \int I(\mathbf{n}) e^{-i2\pi \frac{\mathbf{n} \cdot (\mathbf{x}_2 - \mathbf{x}_1)}{\lambda}} \, d\Omega
\end{equation}

定义基线矢量 $\mathbf{b} = \mathbf{x}_2 - \mathbf{x}_1$,最终结果为:
\begin{equation}
    \langle V_1 V_2^* \rangle = K^2 \Delta \nu \int I(\mathbf{n}) e^{-i2\pi \frac{\mathbf{n} \cdot \mathbf{b}}{\lambda}} \, d\Omega
\end{equation}

\end{document}