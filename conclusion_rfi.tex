\documentclass{article}
\usepackage{amsmath}
\usepackage{physics}
\usepackage[UTF8, fontset=none]{ctex} % 禁用默认字体
\usepackage{fontspec}
\usepackage{graphicx}
\setmainfont{Noto Sans CJK SC}
\usepackage{booktabs}
\usepackage{array}    % 列格式调整
\usepackage{xcolor}   % 颜色控制(可选)
\usepackage{geometry} % 页面边距设置

\title{Summaries of Papers}
\author{Xitan}

\begin{document}
\maketitle
\section{The radio environment of the 21 Centimeter Array:RFI detection and mitigation}
\subsection{研究背景与目标}
\subsubsection{科学意义}
\begin{itemize}
    \item 探测宇宙黎明(\textbf{Cosmic Dawn})和再电离时代(\textbf{EoR})是低频射电天文学的核心目标
    \item 通过中性氢21cm信号(微弱,约$1$-$10\,\mathrm{mK}$)揭示早期宇宙结构演化
\end{itemize}

\subsubsection{技术挑战}
\begin{itemize}
    \item 低频射电观测($50$-$200\,\mathrm{MHz}$)面临强射频干扰(\textbf{RFI})
    \item 干扰源包括:
    \begin{itemize}
        \item 地面人为干扰(FM广播、卫星通信)
        \item 天空源(银河系/河外射电源旁瓣泄漏、栅瓣)
    \end{itemize}
    \item \textbf{RFI抑制}是数据处理的\textbf{首要步骤}
\end{itemize}

\subsection{RFI检测与抑制方法}
\subsubsection{地面RFI处理(标记直接剔除)}
\subsubsection{统计阈值法}
\begin{itemize}
    \item \textbf{原理}:热噪声与宇宙信号服从高斯分布,RFI为非高斯异常
    \item \textbf{实施}:
    \begin{itemize}
        \item 计算频率通道均值$\mu$和标准差$\sigma$
        \item 剔除超过阈值(如$3\sigma$-$7\sigma$)的数据点
        \item 高频污染通道(如$137\,\mathrm{MHz}$卫星信号)采用严格阈值
    \end{itemize}
    \item \textbf{效果}:
    \begin{itemize}
        \item RFI占用率在$115$-$163\,\mathrm{MHz}$为$2.66\%$
        \item 与LOFAR($3.18\%$)、MWA($1.65\%$)相当
    \end{itemize}
\end{itemize}

\subsubsection{可见度相关系数}
\begin{itemize}
    \item \textbf{定义}:
    $$
    p_{ij} = \frac{V_{ij}}{\sqrt{V_{ii}V_{jj}}}
    $$
    \item \textbf{优势}:抑制同时出现在分子分母的强RFI
    \item \textbf{实时处理}:
    \begin{itemize}
        \item 效率降低$22\%$
        \item 数据离散度显著减小
    \end{itemize}
\end{itemize}

\subsection{天空RFI处理}
\begin{itemize}
    \item \textbf{旁瓣泄漏}:21CMA规则天线间距导致
    \begin{itemize}
        \item 主要污染源:天鹅座A(Cygnus A)银道面每天会扫过北天极区域一次,这使得银河系的辐射每天通过两个旁瓣两次进入21CMA的视场。
    \end{itemize}
    \item \textbf{栅瓣}:21CMA规则天线间距导致栅瓣(grating sidelobes)
        \begin{itemize}
            \item 等间距的圆环的空间响应几乎与主瓣相当,如果它们“看
到”银河系和河外射电源,就会对我们的观测造成严重干扰。例如两个强射电源:天鹅座A(Cygnus A)和仙后座A(Cas A)。
            \end{itemize}
    \item \textbf{解决方案}:识别污染时间窗口,结合CLEAN算法建模
\end{itemize}


\subsection{实验结果与验证}
\begin{itemize}
    \item \textbf{灵敏度验证}:系统噪声按$t^{-1/2}$下降
    \item \textbf{数据质量}:
    \begin{itemize}
        \item RFI抑制后可见度功率谱反映银河系辐射轮廓
        \item 经过射频干扰抑制后旁瓣泄漏的两个强射电源任然有影响
    \end{itemize}
\end{itemize}

\subsection{结论}
尽管对rfi的处理和剔除是比较好的,但是没有考虑更复杂的的dde影响,其次由于分辨率、灵敏度不够高对于弱射电源的搜寻和处理都带来了极大困难,并且由于阵列的规则排列对于强射电源会带来傍瓣、栅瓣的影响难以扣除。


\section{Radio Sources in the NCP Region Observed with the 21 Centimeter Array}
\subsection{背景}
本文基于21厘米阵列(21CMA)对北天极(NCP)区域的12小时观测数据,构建了包含624个射电源的目录,覆盖频率75-175 MHz,角分辨率$\sim 4'$。通过限制东西向基线($L < 1500$ m)降低电离层效应影响,源探测灵敏度达0.1 Jy。分析显示源计数在$\sim 1$ Jy以下趋于平缓,与GMRT及MWA观测结果一致。研究发现,校准误差及亮源(如3C061.1)栅瓣是限制成像动态范围($\sim 10^4$)的主要因素。本文结果为SKA等下一代低频干涉仪的设计提供了关键参考。
\subsection{引言}
低频射电观测($\nu \leq 300$ MHz)对研究河外射电统计特性及再电离时期(EoR)信号至关重要。北天极区域因全天候可观测特性,成为21CMA、LOFAR等阵列的重点目标。本文通过21CMA观测数据,首次系统分析NCP区域射电源分布特性,探讨低频干涉成像的技术挑战。

\subsection{方法与观测细节}
\subsubsection{21CMA阵列配置}
\begin{itemize}
    \item \textbf{硬件:} 81个站点,10287个对数周期天线,东西基线6.1 km,南北4 km,固定指向NCP。
    \item \textbf{参数:} 频率50-200 MHz,角分辨率$\sim 3'$(200 MHz),系统噪声温度$\sim 40$ K。
    \item \textbf{数据处理:} GPU加速相关运算,12.5 MHz子波段成像,Hogbom CLEAN反卷积。
\end{itemize}

\subsubsection{关键步骤}
\begin{itemize}
    \item \textbf{RFI抑制:} 结合阈值剔除(3$\sigma$)与高斯统计过滤,保留97\%数据。
    \item \textbf{自校准:(自校准和闭合关系)} 基于10个亮源(如3C061.1,$S_{150\,\text{MHz}} = 33$ Jy)每小时相位校正。
    \item \textbf{成像:}
    \begin{table}[htbp]
\centering
\caption{射电干涉成像处理参数与影响}
\label{tab:radio_imaging}
\begin{tabular}{@{}lll@{}}
\toprule
\textbf{步骤} & \textbf{参数/方法} & \textbf{作用与影响} \\
\midrule
基线筛选 & 
$100\,\text{m} \leq L \leq 1500\,\text{m}$ & 
抑制电离层扰动,优化uv覆盖完整性,降低噪声。 \\
\addlinespace

网格化 & 
$4096^2$网格 + 均匀加权 & 
高分辨率uv采样,但可能引入栅瓣伪影(需反卷积校正)。 \\
\addlinespace

反卷积 & 
Högbom算法(增益0.05) & 
去除点源旁瓣,残留伪影水平依赖迭代深度与终止条件。 \\
\addlinespace

频率合并 & 
$\SI{1.56}{\mega\hertz}$子带 & 
平衡频率分辨率与灵敏度,抑制热噪声($\alpha \Delta \nu^{-1/2}$)。 \\
\addlinespace

视场裁剪 & 
中心$14^\circ \times 14^\circ$ & 
聚焦科学目标区域,避免边缘效应(如波束畸变)。 \\
\bottomrule
\end{tabular}
\end{table}
    
    \item \textbf{波束建模:} 高斯近似主波束$\theta_b = 3.58^\circ (\nu/100\,\text{MHz})^{-1.21}$,验证增益$G \propto \nu^2$。
\end{itemize}

\subsection{关键结果}
\subsubsection{源选择}
\begin{itemize}
    \item 当前方法:要求候选源在至少4个子波段(共8个)中满足峰值流量$S_p$≥5$\sigma$(局部噪声阈值),并结合空间位置一致性验证(偏移≤3′)。
    \item 全子波段放宽法(极端情况):候选源在任意子波段中满足$\ S_p≥3σ$即保留。
    \item 相邻通道关联法:利用旁瓣的频率依赖性(旁瓣位置随频率偏$\propto \nu^{-1}$ ),仅保留在相邻子波段($\Delta \nu = 24.4 kHz$)中同时出现的候选源
\end{itemize}

\subsubsection{源目录与特性}
\begin{itemize}
\item 探测624个射电源(半径$5^\circ$内)
\item 交叉验证:490个源与6C巡天一致($>0.1$ Jy)
\item 光谱指数分布:
  \begin{equation}
    \alpha = -0.8\ (\text{峰值}),\quad \alpha_{\text{高频}} \sim -1.25
  \end{equation}
\end{itemize}

\subsubsection{源计数特性}
\begin{equation}
\frac{dN}{dS} = 
\begin{cases}
k_1 S^{-\gamma_1}, & S < S_0 \\
k_2 S^{-\gamma_2}, & S \geq S_0
\end{cases}
\end{equation}
其中$\gamma_1=1.59$(GMRT),$\gamma_2=2.1-2.55$。

\subsection{噪声分析}
我们现在分析朝向北天极(NCP)区域的干涉成像中的噪声,主要有四个来源:热噪声、混淆噪声、校准误差和解卷积误差。
\begin{itemize}
    \item 热噪声: $$s_{the} = \frac{ \sqrt{2} k_B T_{sys}} {A_{eff} \eta \sqrt{\Delta\nu \Delta t}}$$
    \item 混淆噪声:
    \begin{equation}
    \sigma_{c}^{2} = \Omega_{b} \int_{0}^{S_{\text{lim}}} \frac{dN}{dS} S^{2} \, dS
    \label{eq:sigma_squared}
    \end{equation}
    \item 校准误差和解卷积误差: 事实上,$S_{lim}$除了受望远镜分辨率影响外,还受校准误差和反卷积误差的共同控制。目前的成像算法尚未考虑电离层效应,不过如果将时间步长选择在几分钟以内,或者仅使用短基线,自校准可能会部分校正该效应。在几分钟的时间尺度上进行校准非常耗时,因为还需要在每个频率通道上进行校正。此外,21CMA(21厘米射电阵列)众多冗余基线产生的栅瓣,会在像B004713+891245和3C061.1 这样的亮源周围形成非常明显、等间距的圆环。目前,在反卷积处理中完全去除这些环状结构仍然很困难。
\end{itemize}

\subsection{讨论与启示}
\subsubsection{技术挑战}
\begin{itemize}
\item 亮源处理:3C061.1残留栅瓣导致$10^4$动态范围限制
\item 弱源处理:分辨率的限制导致的混淆效应,导致的源计算不完整性,同时对于扩展源的识别和处理也极具挑战 
\item 电离层扰动:短基线($L<1500$ m)相位误差公式:
  \begin{equation}
    \Delta\phi \propto L\cdot\nabla N_e\cdot\nu^{-1}
  \end{equation}
\end{itemize}

\subsubsection{对SKA的启示}
\begin{table}[ht]
\centering
\caption{下一代阵列设计建议}
\label{tab:skalow}
\begin{tabular}{@{}ll@{}}
\toprule
参数 & 推荐方案 \\
\midrule
基线布局 & 混合长短基线(随机分布抑制栅瓣) \\
校准策略 & 冗余基线+实时GNSS TEC监测 \\
成像算法 & w-term aware CLEAN \\
\bottomrule
\end{tabular}
\end{table}

\subsection{结论}
成功构建NCP射电源目录,验证低频源计数平缓趋势,当前限制因素:校准误差(占噪声贡献68\%)和反卷积残留,目前电离层效应和亮扩展源带来的校准误差很难处理,以及天区内和天区外强射电源的栅瓣和旁瓣。由于灵敏度不够高识别微弱射电源困难导致的混淆效应。

\end{document}